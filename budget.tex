Jednym z kryteriów, które przyjęliśmy w trakcie projektowania robota była jego cena, która nie powinna przekroczyć 30 zł/os. W związku z tym staraliśmy się maksymalnie ciąć koszty, szukając u siebie i wśród znajomych części, które mogłyby przydać się do zbudowania robota - szczególnie koncentrując się na silniku i zasilaniu. Niestety, zarówno zasilanie, jak i silnik okazały się zbyt mało wydajne i zostaliśmy zmuszeni do zakupu nowych.

Zdecydowaliśmy się także na wspieranie lokalnego przemysłu elektronicznego, kupując części w sklepach elektronicznych w całym Poznaniu, zamiast korzystając z ofert dostępnych w internecie. Dzięki temu udało nam się oszczędzić na ewentualnych kosztach wysyłki.

Nie byliśmy w stanie wycenić niektórych elementów, które miały niebagatelny wpływ na Papaja. Mamy tu na myśli szczególnie korki do wina, których wartość niełatwo ustalić bez ich powiązania z butelkami od wina (niekoniecznie pustymi). 

Łączny koszt robota wyniósł 168,6 zł, co w przeliczeniu na osobę daje niewiele ponad 24 polskich nowych złotych. Jest to naszym zdaniem bardzo udany wynik.

\begin{table}[!htbp]
\centering
\caption{Elementy wykorzystane w Papaju (w zł)}
\label{table:koszta}
\begin{tabular}{l|r|r|r}
Nazwa                          & Cena & Ilość & Koszt \\
\hline
środek drobnokrystaliczny B327 & 8    & 1     & 8     \\
laminat jednostronny 190x3     & 15   & 1     & 15    \\
LEDy 3mm                       & 0,2  & 3     & 0,6   \\
sterownik silnika L293D        & 11   & 1     & 11    \\
potencjometr drutowy           & 4,9  & 1     & 4,9   \\
gniazdo jednorzędowe BLS-09    & 0,45 & 2     & 0,9   \\
BL-T piny do gniazd BLS        & 0,1  & 24    & 2,4   \\
gniazdo jednorzędowe BLS-02    & 0,1  & 2     & 0,2   \\
podstawka DIL-28               & 0,5  & 1     & 0,5   \\
zestaw przewodów               & 7    & 1     & 7     \\
płytka prototypowa SD12NB      & 14   & 1     & 14    \\
listwa dwurzędowa PLD40        & 1,5  & 1     & 1,5   \\
stabilizator napięcia L-7805C  & 1,9  & 1     & 1,9   \\
transoptory CNY70              & 3    & 7     & 21    \\
rezonator kwarcowy             & 1,8  & 1     & 1,8   \\
dławik 12uH                    & 0,2  & 1     & 0,2   \\
LEDy 5mm                       & 0,4  & 5     & 2     \\
silnik + koło                  & 17   & 2     & 34    \\
listwa jednorzędowa PLS-40     & 1,4  & 2     & 2,8   \\
papier kredowy                 & 0,6  & 1     & 0,6   \\
zestaw 8 baterii AA            & 15   & 1     & 15    \\
komparatory                    & 0,7  & 2     & 1,4   \\
koszyki                        & 2,5  & 2     & 5     \\
ATmega8A-PU                    & 6    & 1     & 6     \\
\hline
łącznie                          &      &       & 157,7
\end{tabular}

\centering
\caption{Wykorzystane rezystory (w zł)}
\begin{tabular}{l|r|r|r}
Nazwa          & Koszt & Ilość & Suma \\
\hline
metaloxid 150$\Omega$  & 0,2 & 3 & 0,6 \\
metaloxid 100k$\Omega$  & 0,2 & 7 & 1,4 \\
węglowy 10k$\Omega$     & 0,3 & 8 & 2,4 \\
węglowy 240$\Omega$     & 0,3 & 7 & 2,1 \\
\hline
łącznie           &     &   & 6,5
\end{tabular}

\caption{Wykorzystane kondensatory (w zł)}
\begin{tabular}{l|r|r|r}
Nazwa          & Koszt & Ilość & Suma \\
\hline
ceram. dyskowy 100 nF & 0,2 & 12 & 2,4 \\
elektrolityczny 10 uF & 0,5 & 2  & 1   \\
elektrolityczny 47 uF & 0,2 & 2  & 0,4 \\
elektrolityczny 22 uF & 0,2 & 1  & 0,2 \\
ceramiczny 22 pF      & 0,2 & 2  & 0,4 \\
\hline
łącznie                &     & & 4,4
\end{tabular}
\end{table}
