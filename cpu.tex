Sercem robota jest mikrokontroler ATmega8A-PU. Zdecydowaliśmy się na niego ze względu na cenę, łatwość programowania oraz jego popularność, co miało szczególnie znaczenie podczas problemów z działaniem robota. Doskonała dokumentacja do ATmegi i wiele rozwiązań powszechnie dostępnych w Sieci zdecydowanie ułatwiło nam okiełznanie Papaja i zabezpieczenie go przed niepożądanym działaniem. Poza tym uważamy, że Arduino jest dla słabych i jego wybór mógłby być drogą na skróty, co uwłaszczałoby naszej godności, honorowi i rozumowi człowieka.

Utwierdziliśmy się w przekonaniu, że nasza decyzja była słuszna, gdy okazało się, że naszym wyborem zainspirowało się wiele grup występujących w~konkursie RoboDay. Jedna z nich nawet wylądowała na podium, zatem możemy z~całą pewnością stwierdzić, że jest to też nasz sukces. Można nawet zaryzykować stwierdzenie, że de facto zajęliśmy dwa miejsca na podium, co wprawia nas w~zadumę i~zachęca do dalszej pracy.


