\documentclass{article}
\usepackage{polski}
\usepackage[utf8]{inputenc}
\usepackage{indentfirst} 
\newcommand{\HRule}{\rule{\linewidth}{0.5mm}}

\begin{document}

\begin{titlepage}
\begin{center}

% Upper part of the page. The '~' is needed because \\
% only works if a paragraph has started.

\textsc{\LARGE Politechnika Poznańska}\\[1.5cm]

\textsc{\Large Dokumentacja techniczno-rozruchowa}\\[0.5cm]

% Title
\HRule \\[0.4cm]
{ \huge \bfseries PAPAJ \\[0.4cm] }
\textsc{\Large robot typu line-follower}\\[0.5cm]

\HRule \\[1.5cm]

% Author and supervisor
\noindent
\begin{minipage}[t]{0.4\textwidth}
\begin{flushleft} \large
\emph{Autorzy:}\\
Łukasz \textsc{Kosiak} (c)\\
Łukasz \textsc{Królik} \\
Piotr \textsc{Kurzawa} \\
Patryk \textsc{Łapiezo} \\
Sebastian \textsc{Pawlak} \\
Bartosz \textsc{Sławianowski} \\
Kacper \textsc{Stępień} \\
Pedro \textsc{Brêda}
\end{flushleft}
\end{minipage}%
\begin{minipage}[t]{0.4\textwidth}
\begin{flushright} \large
\emph{Prowadzący:} \\
dr. inż. Rafał \textsc{Klaus}
\end{flushright}
\end{minipage}

\vfill

% Bottom of the page
{\large \today}

\end{center}
\end{titlepage}
\tableofcontents
\newpage

\section{Opis}

Papaj jest robotem typu line-follower stworzonym na potrzeby corocznego konkursu RoboDay na Politechnice Poznańskiej, wykonanym przez zdolnych i dumnych studentów z grupy I4.2. 

Nasze urządzenie jest klasycznym przykładem line-followera. Posiada on napęd na tylne koła (każde posiada własny silnik), jest wyposażony w ślizgacze w przedniej części robota oraz 7 czujników linii. Całość jest zasilana popularnymi "paluszkami", co przekłada się na niską cenę.

Papaj został sensacyjnym zwycięzcą RoboDay 2015 w kategorii line-followerów zostawiając w tyle bardziej rozbudowane i droższe konstrukcje.

\section{Nazwa}
				
Robocza nazwa Papaja brzmiała "Pommes frites" od napisu, który znajdował się na pudełku, który robił za miejsce jego przechowywania - jednak z powodu niedawnych obchodów 95-rocznicy urodzin Wielkiego Polaka Jana Pawła II, postanowiliśmy nazwać naszego robota na jego cześć. 

Tym samym dołożyliśmy małą cegiełkę w rozpowszechnianie jedynej słusznej prawdy o naszym papieżu. 
						
\section{Budżet}

Jednym z kryteriów, które przyjęliśmy w trakcie projektowania robota była jego cena, która nie powinna przekroczyć 30 zł/os. W związku z tym staraliśmy się maksymalnie ciąć koszty, szukając u siebie i wśród znajomych części, które mogłyby przydać się do zbudowania robota - szczególnie koncentrując się na silniku i zasilaniu. Niestety, zarówno zasilanie, jak i silnik okazały się zbyt mało wydajne i zostaliśmy zmuszeni do zakupu nowych.

Trzeba również zaznaczyć, że nie wszystkie elementy zostały ostatecznie użyte do zbudowania robota (wiele z nich zostało kupione "na zapas") - jednak nie mają one większego wpływu na łączny koszt naszego line-followera.

\begin{table}[c]
\caption{Spis części zakupionych na potrzeby Papaja (ceny w zł)}
\label{table:koszta}
\centering
\begin{tabular}{l|r|r|r}
Element                       & Koszt jedn. & Ilość & Koszt łączny \\
\hline
Transaptor odbiciowy CNY70    & 2,00        & 7     & 14,00        \\
Papier kredowy                & 0,60        & 1     & 0,60         \\
Środek drobnokrystaliczny     & 8,00        & 1     & 8,00         \\
Laminat jednostronny 190x3    & 15,00       & 1     & 15,00        \\
Kondensator ceram. dyskowy    & 0,20        & 12    & 2,40         \\
LED 3mm                       & 0,20        & 3     & 0,60         \\
Sterownik silnika L-293D      & 11,00       & 1     & 11,00        \\
Dławik AL0307                 & 0,20        & 1     & 0,20         \\
Rezystory metaloxide          & 0,20        & 18    & 3,60         \\
20K 0,2W                      & 4,90        & 1     & 4,90         \\
Gniazdo jednorzędowe BLS-09   & 0,45        & 2     & 0,90         \\
Piny do gniazd BLS BL-T       & 0,10        & 24    & 2,40         \\
Gniazda jednorzędowe BLS-02   & 0,10        & 2     & 0,20         \\
Podstawka DIL-28              & 0,50        & 1     & 0,50         \\
Zestaw przewodów              & 7,00        & 1     & 7,00         \\
Płytka stykowa SD12NB         & 14,00       & 1     & 14,00        \\
Listwa 2-rzędowa PLD40        & 1,50        & 1     & 1,50         \\
Bateria 9V                    & 2,60        & 1     & 2,60         \\
BC-639 NPN 100V               & 0,40        & 2     & 0,80         \\
Zacisk do baterii 9V          & 1,60        & 1     & 1,60         \\
Stabilizator napięcia L-7805C & 1,90        & 1     & 1,90         \\
CMOS-4051                     & 1,50        & 1     & 1,50         \\
Rezystory węglowe             & 0,30        & 2     & 0,60         \\
Kondensator 10uF 16V          & 0,50        & 5     & 2,50         \\
LED 5mm                       & 0,40        & 5     & 2,00         \\
Łącznik refl. CNY70           & 5,50        & 1     & 5,50         \\
Kondensator ele. 47uF 16V     & 0,20        & 2     & 0,40         \\
Rezonator kwarcowy 0-16MHz    & 1,80        & 1     & 1,80         \\
Kondesator ele. 22uF 16V      & 0,20        & 2     & 0,40         \\
Dławik radialny 10uH          & 1,30        & 1     & 1,30         \\
ATmega8A-PU DIP28             & 6,99        & 3     & 20,97        
\end{tabular}
\end{table}


Dokładny spis elementów zakupionych na potrzeby Papaja znajduje się w tabeli \ref{table:koszta} na stronie \pageref{table:koszta}.

\section{Schemat ideowy}

\subsection{Jednostka centralna}

\subsection{Zasilanie}

\subsection{Sterowanie silnikami}

\subsection{Czujniki odbiciowe}

\section{Mechanika}

\subsection{Płytka}

\subsection{Układ jezdny}

\subsection{Model 3D}

\section{Kod źródłowy}

[piezol here]

\section{Wnioski}

\section{Podziękowania}

Na wstępie chcielibyśmy podziękować naszemu prowadzącemu, dr. inż. Rafałowi Klausowi,  za nieocenione wsparcie i wkład w budowę naszego robota. Jesteśmy dumni ze współpracy z dr. Klausem i mamy cichą nadzieję, że również on jest z naszej pracy dumny.

Dziękujemy także Maciejowi Uniejewskiemu i Mateuszowi Sarbinowskiemu, którzy  wytykając nam brak ambicji i odrzucając naszą propozycję współpracy, zachęcili nas do tytanicznej pracy zakończonej pełnym sukcesem. Z litości nie będziemy wspominać, jakie pozycje w konkursie zajęły grupy, do których dwie powyższe jednostki należały. Mamy również nadzieję, że taka sytuacja w przyszłości się nie powtórzy - choć obiecujemy, że tego Wam nigdy nie zapomnimy.

Nie możemy zapomnieć również o Janie Pawle II, który dawał nam inspirację i chęci do tworzenia robota, pomimo wszelkich przeciwności losu.

Na koniec pragniemy serdecznie podziękować grupie tworzącej robota "Mały, ale wariat" z Aleksandrą Kobus na czele. Doprawdy rzadko zdarza się sytuacja, aby grupa, która sama tworzyła trasę finalną i miała przywilej testowania swojego robota na swojej trasie parę dni przed zawodami, zawody te przegrała. Szczerze gratulujemy.

\end{document}