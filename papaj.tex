\documentclass{article}
\usepackage{polski}
\usepackage[utf8]{inputenc}
\usepackage{indentfirst} 
\usepackage{graphicx} 
\newcommand{\HRule}{\rule{\linewidth}{0.5mm}}

\begin{document}

\begin{titlepage}
\begin{center}

% Upper part of the page. The '~' is needed because \\
% only works if a paragraph has started.

\textsc{\LARGE Politechnika Poznańska}\\[1.5cm]

\textsc{\Large Dokumentacja techniczno-rozruchowa}\\[0.5cm]

% Title
\HRule \\[0.4cm]
{ \huge \bfseries PAPAJ \\[0.4cm] }
\textsc{\Large robot typu line-follower}\\[0.5cm]

\HRule \\[1.5cm]

% Author and supervisor
\noindent
\begin{minipage}[t]{0.4\textwidth}
\begin{flushleft} \large
\emph{Autorzy:}\\
Łukasz \textsc{Kosiak} (c)\\
Łukasz \textsc{Królik} \\
Piotr \textsc{Kurzawa} \\
Patryk \textsc{Łapiezo} \\
Sebastian \textsc{Pawlak} \\
Bartosz \textsc{Sławianowski} \\
Kacper \textsc{Stępień} \\
Pedro \textsc{Brêda}
\end{flushleft}
\end{minipage}%
\begin{minipage}[t]{0.4\textwidth}
\begin{flushright} \large
\emph{Prowadzący:} \\
dr inż. Rafał \textsc{Klaus}
\end{flushright}
\end{minipage}

\vfill

% Bottom of the page
{\large \today}

\end{center}
\end{titlepage}

\tableofcontents
\newpage

\section{Opis}

Papaj jest robotem typu line-follower stworzonym na potrzeby corocznego konkursu RoboDay na Politechnice Poznańskiej, wykonanym przez zdolnych i dumnych studentów z grupy I4.2. 

Nasze urządzenie jest klasycznym przykładem line-followera. Posiada on napęd na tylne koła (każde z nich posiada własny silnik), jest wyposażony w ślizgacze w przedniej części robota oraz 7 czujników linii. Całość jest zasilana popularnymi "paluszkami", co przekłada się na niską cenę.

Papaj został sensacyjnym zwycięzcą RoboDay 2015 w kategorii line-follower zostawiając w tyle bardziej rozbudowane i droższe konstrukcje.

\section{Nazwa}
				
Robocza nazwa Papaja brzmiała "Pommes frites" od napisu, który znajdował się na pudełku, który robił za miejsce jego przechowywania - jednak z powodu niedawnych obchodów 95-rocznicy urodzin Wielkiego Polaka Jana Pawła II, postanowiliśmy nazwać naszego robota na jego cześć. 

Tym samym dołożyliśmy małą cegiełkę w rozpowszechnianie jedynej słusznej prawdy o naszym papieżu. 
						
\section{Budżet}

Jednym z kryteriów, które przyjęliśmy w trakcie projektowania robota była jego cena, która nie powinna przekroczyć 30 zł/os. W związku z tym staraliśmy się maksymalnie ciąć koszty, szukając u siebie i wśród znajomych części, które mogłyby przydać się do zbudowania robota - szczególnie koncentrując się na silniku i zasilaniu. Niestety, zarówno zasilanie, jak i silnik okazały się zbyt mało wydajne i zostaliśmy zmuszeni do zakupu nowych.

Trzeba również zaznaczyć, że nie wszystkie elementy zostały ostatecznie użyte do zbudowania robota (wiele z nich zostało kupione "na zapas") - jednak nie mają one większego wpływu na łączny koszt naszego line-followera.

\begin{table}[c]
\caption{Spis części zakupionych na potrzeby Papaja (ceny w zł)}
\label{table:koszta}
\centering
\begin{tabular}{l|r|r|r}
Element                       & Koszt jedn. & Ilość & Koszt łączny \\
\hline
Transaptor odbiciowy CNY70    & 2,00        & 7     & 14,00        \\
Papier kredowy                & 0,60        & 1     & 0,60         \\
Środek drobnokrystaliczny     & 8,00        & 1     & 8,00         \\
Laminat jednostronny 190x3    & 15,00       & 1     & 15,00        \\
Kondensator ceram. dyskowy    & 0,20        & 12    & 2,40         \\
LED 3mm                       & 0,20        & 3     & 0,60         \\
Sterownik silnika L-293D      & 11,00       & 1     & 11,00        \\
Dławik AL0307                 & 0,20        & 1     & 0,20         \\
Rezystory metaloxide          & 0,20        & 18    & 3,60         \\
20K 0,2W                      & 4,90        & 1     & 4,90         \\
Gniazdo jednorzędowe BLS-09   & 0,45        & 2     & 0,90         \\
Piny do gniazd BLS BL-T       & 0,10        & 24    & 2,40         \\
Gniazda jednorzędowe BLS-02   & 0,10        & 2     & 0,20         \\
Podstawka DIL-28              & 0,50        & 1     & 0,50         \\
Zestaw przewodów              & 7,00        & 1     & 7,00         \\
Płytka stykowa SD12NB         & 14,00       & 1     & 14,00        \\
Listwa 2-rzędowa PLD40        & 1,50        & 1     & 1,50         \\
Bateria 9V                    & 2,60        & 1     & 2,60         \\
BC-639 NPN 100V               & 0,40        & 2     & 0,80         \\
Zacisk do baterii 9V          & 1,60        & 1     & 1,60         \\
Stabilizator napięcia L-7805C & 1,90        & 1     & 1,90         \\
CMOS-4051                     & 1,50        & 1     & 1,50         \\
Rezystory węglowe             & 0,30        & 2     & 0,60         \\
Kondensator 10uF 16V          & 0,50        & 5     & 2,50         \\
LED 5mm                       & 0,40        & 5     & 2,00         \\
Łącznik refl. CNY70           & 5,50        & 1     & 5,50         \\
Kondensator ele. 47uF 16V     & 0,20        & 2     & 0,40         \\
Rezonator kwarcowy 0-16MHz    & 1,80        & 1     & 1,80         \\
Kondesator ele. 22uF 16V      & 0,20        & 2     & 0,40         \\
Dławik radialny 10uH          & 1,30        & 1     & 1,30         \\
ATmega8A-PU DIP28             & 6,99        & 3     & 20,97        
\end{tabular}
\end{table}


Dokładny spis elementów zakupionych na potrzeby Papaja znajduje się w tabeli \ref{table:koszta} na stronie \pageref{table:koszta}.

\section{Schemat ideowy}

\subsection{Jednostka centralna}

\subsection{Zasilanie}

\subsection{Sterowanie silnikami}

\subsection{Czujniki odbiciowe}

\section{Mechanika}

\subsection{Płytka}
Centralnym elementem, łączącym komponenty robota w całość jest drukowana płytka PCB (Printed Circuit Board). Jej ścieżki zastąpić miały dziesiątki kabli i przewodów łączących elementy. Proces budowy takiej płytki nie jest trudny i skomplikowany, pod warunkiem oczywiście, że człowiek wykona swoją pracę starannie, dokładnie i z pełnym oddaniem.

\subsubsection{Projektowanie}

Płytkę należało zaprojektować w odpowiednio dla tego celu stworzonym programie. Początkowo nasza płytka była tworzona w słynnym EAGLE'u, ostatecznie jednak porzuciliśmy go na rzecz KiCADa, ponieważ jest to narzędzie open-source, a takie oprogramowanie jest szczególnie bliskie naszemu sercu. Niestety, mimo naszej niechęci do zamkniętego oprogramowania nadal byliśmy zmuszeni skorzystać z systemu operacyjnego Windows.  

Projekt ten musiał być wykonany w odbiciu lustrzanym, tak, aby po odbiciu obrazu na płytkę, była ona w prawidłowym położeniu. Aby ułatwić sobie nieco sprawę, umieściliśmy na płytce napis „AKD’15 WI PP” ku chwale Akademii Kreatywnego Działania dr Rafała Klausa.

Na wyjściu programu otrzymaliśmy jednostronną płytkę PCB. Niestety sieć połączeń nie zapewniała możliwości poprowadzenia naraz wszystkich połączeń bezkolizyjnie, dlatego też potrzebne było poprowadzenie paru połączeń dodatkowymi kablami. Dzięki temu udało się uniknąć konieczności projektowania znacznie bardziej skomplikowanej konstrukcyjnie płytki dwustronnej. 

\begin{figure}
\centering
\includegraphics[scale=0.75]{board.png}
\caption{Projekt płytki}
\end{figure}
\subsubsection{Drukowanie}

Profesjonaliści biegli w konstruowaniu płytek dysponują sprzętem, który umożliwia konstruowanie płytek o wielu warstwach. Sprzętem takim dysponuje również nasza politechnika, lecz jak się okazało, korzystanie z podobnego sprzętu zostało zakazane. Zostaliśmy zobligowani do metody chałupniczej, tzw. żelazkowej, profesjonalnie nazywaną metodą termotransferową. Drukowanie postępowało w następujących etapach: 

\afterpage{%
\thispagestyle{empty}
\begin{figure}[!htbp]
\centering
\includegraphics[scale=0.5]{gora.png}
\caption{Płytka PCB - rzut górny}
\includegraphics[scale=0.45]{dol.png}
\caption{Płytka PCB - rzut dolny}
\end{figure}
}

\begin{itemize}
\item Wydrukować drukarką laserową, w możliwie jak najlepszej jakości, obraz płytki powstałego w wyniku realizacji punktu 1. Preferowanym typem papieru jest tzw. papier kredowy, którego przewaga nad zwykłym papierem jest taka, że  jego powierzchnia jest bardzo śliska, dzięki czemu słabiej doń przywiera toner.
\item Dociąć laminat do rozmiaru wydrukowanego obrazu. Jeśli nie jest on pierwszy raz drukowany, należy go bardzo dokładnie umyć z resztek toneru oraz innych zabrudzeń.
\item Przyłożyć papier kredowy do laminatu i dokładnie nałożyć papier, brzegi zaś kartki nie będące zadrukowane tonerem, użyć do przyklejenia do kartki. Zwrócić uwagę na fałdki papieru, brudy oraz inne podobne defekty.
\item Ogrzewać płytki rozgrzanym żelazkiem. Ma ono być możliwie jak najcieplejsze. Dokładnie zgrzać całą powierzchnię kartki, zwracając również uwagę na brzegi kartki, zazwyczaj traktowane po macoszemu. Czas trwania kroku powinien wynosić ok. 15-20 minut. W wyniku okazać się ma, że cały toner został odklejony z papieru i odbił się na miedzianej stronie laminatu.
\item Włożyć płytkę do wody. Można dodać parę kropel detergentu. Płytkę trzymać w wodzie przez około 15 minut - dzięki temu papier zostanie dobrze odmoczony i będzie lepiej odchodzić od tonera. Po odmoczeniu płytki odkleić taśmę klejącą i zdjąć kartkę. Jeśli w wyniku tego nie odejdzie większa część obrazu, można przejść do kolejnego punktu, w przeciwnym razie należy powtórzyć całą procedurę powtórnie. Nasza grupa była zmuszona do trzykrotnego drukowania płytki na nowo. Nie należy zatem zrażać się niepowodzeniami, bo jest to klucz do sukcesu.
\item Ostrożnie, z wyczuciem odpowiednim ścierać palcem resztki papieru - TYLKO W MIEJSCACH KTÓRE NIE SĄ POKRYTE CZARNYM TONEREM. Ułatwieniem jest z pewnością to, że białe miejsca nie są do niczego przyklejone, dzięki czemu teoretycznie łatwiej odchodzą. Należy co jakiś czas wkładać ponownie płytkę do wody, z powodu, że woda co jakiś czas znika z powierzchni płytki - głównie przez kontakt z dłońmi oraz w wyniku parowania. Należy uważnie odsłonić ścieżki, tak, że będą one błyszczeć miedzią, z której w rzeczywistości są zbudowane. Jest to najżmudniejsza część metody termotransferowej (nam zajęła około 2~godzin).
\end{itemize}

Metoda ta jest przyczyną bólu rąk, dlatego też ważne jest by działać w~zespole oraz wykazać się kreatywnością w konstrukcji narzędzi, np. małego rylca do zdzierania resztek papieru. Pomocą w takim wypadku jest Sz. P. Antonio Vivaldi, którego muzyka pomogła nam przejść ten uciążliwy proces. Serdecznie polecamy jego utwory młodym elektrotechnikom.

\subsubsection{Trawienie}

Należy znaleźć odpowiednie naczynie, które umożliwiało zanurzenie w nim płytki w całości. Naczynie winno mieć odpowiednie parametry mechaniczno-chemiczne, takie jak: brak zewnętrznych otworów umożliwiające dyfuzję płynu trawiącego z otoczeniem (np. dziury) czy odporność na rozpuszczenie.

Nalewamy do naczynia wody o temperaturze około 50 stopni. Do tej wody wsypujemy proszek wytrawiacza i mieszamy (nie palcem, ale szczoteczka do zębów się nada). Wkładamy laminat do kwasu i~czekamy patrząc co się dzieję, można też delikatnie przecierać powierzchnię szczoteczką do zębów. Woda zacznie się robić turkusowa (utleniona miedź). Jeśli temperatura roztworu będzie spadać, można ją podgrzać w garnku, a następnie znowu zalać laminat. 

Zwykle trawienie trwa od 15-30 minut, w zależności od stężenia kwasu i~temperatury. Kiedy wytrawi się miedź dookoła farby i wydaje nam się że jest gotowa, wyciągamy płytkę z kwasu. Następnie delikatnie myjemy wacikiem namoczonym w rozpuszczalniku farbę. Naszym oczom ukaże się piękna ścieżka z~miedzi. Myjemy ją mydłem z rozpuszczalnika i suszymy.

\subsubsection{Wiercenie i lutowanie}

Po operacji trawienia płytka była gotowa do wiercenia. Została ona zatem potraktowana wiertłem 0,7 mm umieszczonym w wiertarce udostępnionej nam przez naszego mentora, dr Klausa. 

Ponieważ wiercenie przebiegło bez większych problemów, natychmiast podjęliśmy się lutowania elementów na płytce. Najpierw płytka została posmarowana roztworem denaturatu i kalafonii - dzięki temu rozwiązaniu cyna łatwiej przylega do płytki. Do lutowania użyliśmy ambitnego sprzętu znanym powszechnie jako stacja lutownicza o regulowanej temperaturze z wykorzystaniem cyny 0,8 mm.

Teoretycznie lutowanie powinien ułatwić fakt, że mieliśmy do czynienia wyłącznie z elementami przelotowymi (tj. posiadającymi przewody, które są potem wkładane w otwory przelotowe na płytce drukowanej i lutowane), jednak spotkaliśmy się z paroma istotnymi problemami:

\begin{itemize}
\item masa zworek (niepożądanych zwarć pomiędzy ścieżkami), które trzeba było usunąć;
\item zimne luty (które mogły powodować przerwanie obwodu i wymagały poprawy),
\item wszystkie ścieżki trzeba było mozolnie testować miernikiem,
\item gorąca kalafonia nam się wylała, co spowodowało, że dostaliśmy tylko 4.5 za wygląd Papaja;
\item brak doświadczenia, co skutecznie wydłużyło prace.
\end{itemize}

Z powyższych powodów koszmar zwany lutowaniem skończył się po mniej więcej dwóch dniach. Jeżeli tak powstaje Chocapic, to współczujemy firmie Nestlé. 

Niemniej było to dla nas miłe doświadczenie i uważamy, że było warto, bo człowiek po takiej działalności czuje dobrze.

\subsubsection{Uwagi}

Często zdarza się, że palec producenta płytki jest zbyt szorstki (tak że zrywa większe połacie tonera) lub jego oko jest na tyle nieuważne, że nie dostrzegł że w którymś miejscu płytki ścieżka jest pokryta papierem. Nie jest to powód do tego, aby ulec załamaniu nerwowemu oraz innym objawom głębokiej depresji.

W przypadku zdarcia czarnych elementów obrazu płytki, należy zaopatrzyć się w możliwie jak najcieńszy flamaster, o właściwościach zapewniających odporność na wodę. Właściwość ta daje ogromne szanse na to, że również ów tusz oprze się wytrawiaczowi. Należy wówczas dorysować zerwane części obrazu. W przypadku jednak, gdyby metoda ta nie pomogła, należy poprowadzić kabel zwykły między brakującymi ścieżkami. 

Innym defektem jest połączenie się ścieżek, w wyniku niedbałego ściągnięcia papieru z powierzchni płytki. Wówczas może dojść do zwarcia, czyli sytuacji, gdzie przewody nie stykają się w kontrolowany sposób, a raczej gdzieś pośrodku. Należy problem ten rozwiązać poprzez przecięcie ścieżek nożem. 


\subsection{Układ jezdny}
Napęd robota składa się z dwóch silniczków. Ich działanie opiera się na metodzie różnicowej, stosowanej chociażby w czołgach. Każde koło jest oddzielnie napędzane i skręt wynika z tego, że jedno koło kręci się szybciej od drugiego. Ruch kółek czerpie się z działania silniczków elektrycznych, których szybkość jest zależna od ilości napięcia jakie zostanie dostarczone na styki. Konieczną procedurą było zatem sprawienie, że ruch silniczka miałby wpływ na ruch kółka. Należało skonstruować jakiś mechanizm przeniesienia jednego ruchu na drugi. 

„Należało”- słowo to padło nie przypadkowo, ponieważ zaniechaliśmy konstrukcji własnej roboty. Byłby to mechanizm niedoskonały, który znacznie spowolniłby robota z powodu niedoskonałości technicznych układu przełożenia. Z tego też powodu postanowiliśmy zainwestować więcej pieniędzy w silniczki, które miało gotową zębatkę dopasowaną do dołączonego kółka. Układ ten bardzo dobrze się sprawdził, nie „gubił” kroków i dzięki temu robot nasz mógł poruszać się tempem bardzo szybkim.

Najprostsza jednak wersja napędu opiera się na przekładni pasowej zbudowanej z popularnej gumki recepturki. Daje ona rezultaty, jest do tego opcją budżetową i ekonomiczną, jednakże ma mały współczynnik tarcia statycznego i nie może osiągnąć zachwycającej prędkości. Rozwiązanie dylematu prędkość vs. koszty zostawiamy Czytelnikowi, którego bardzo serdecznie pozdrawiamy. Osoby posiadające zabawki mogą się również postarać o sprawdzenie, czy z ich komponentów nie dałoby się stworzyć napędu. Przydatne byłby wszelkiego rodzaju zębatki. Ważna jest kreatywność, otwarty umysł i wiara we własne możliwości. 


\subsection{Model 3D}

\section{Kod źródłowy}

[piezol here]

\section{Wnioski}

\section{Podziękowania}

Na wstępie chcielibyśmy podziękować naszemu prowadzącemu, dr. inż. Rafałowi Klausowi,  za nieocenione wsparcie i wkład w budowę naszego robota. Jesteśmy dumni ze współpracy z dr. Klausem i mamy cichą nadzieję, że również on jest z naszej pracy dumny.

Dziękujemy także Maciejowi Uniejewskiemu i Mateuszowi Sarbinowskiemu, którzy  wytykając nam brak ambicji i odrzucając naszą propozycję współpracy, zachęcili nas do tytanicznej pracy zakończonej pełnym sukcesem. Mamy również nadzieję, że taka sytuacja w przyszłości się nie powtórzy - choć obiecujemy, że tego Wam nigdy nie zapomnimy. Z litości nie będziemy wspominać, jakie pozycje w konkursie zajęły grupy, do których dwie powyższe jednostki należały.

Nie możemy zapomnieć również o Janie Pawle II, który dawał nam inspirację i chęci do tworzenia robota, pomimo wszelkich przeciwności losu.

Na koniec pragniemy serdecznie podziękować grupie tworzącej robota "Mały, ale wariat" z Aleksandrą Kobus na czele. Doprawdy rzadko zdarza się sytuacja, aby grupa, która sama tworzyła trasę finalną i miała przywilej testowania swojego robota na swojej trasie parę dni przed zawodami, zawody te przegrała. Szczerze gratulujemy.

\end{document}