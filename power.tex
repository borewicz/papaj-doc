Zasilanie było zdecydowanie największym wyzwaniem podczas budowy Papaja. Pod uwagę były brane najróżniejsze opcje, łącznie z wykorzystaniem używanych akumulatorków pochodzących z telefonów marki Nokia. Zastosowane rozwiązanie, które walnie przyczyniło się do naszego zwycięstwa, przyszło do nas z najmniej spodziewanej strony - sklepu niejakiego Gembary, który zasugerował nam rozwiązanie "tanio i tanio".

Papaj zasilany jest ośmioma popularnymi paluszkami (baterie AA) połączonymi szeregowo, co zdecydowanie zmniejszyło koszty, zapewniało większą elastyczność i dawało nam całe 12V do wykorzystania. Rozwiązanie to ma jednak dużą wadę - znacząco zwiększa masę robota, co w połączeniu z umiejscowieniem baterii w tylnej części robota powodowało, że Papaj podnosił się i rwał ku radości całego zespołu. 

W celu stabilizowania napięcia dla elektroniki na poziomie 5V zastosowany został stabilizator L-7805C i dał radę.  Silniki natomiast delektowały się pełnym napięciem z baterii wynoszącym 12V.